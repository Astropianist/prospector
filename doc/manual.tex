\begin{document}

\section{Summary}

\section{Installation \& Requirements}
You will need:

\begin{itemize}
\item \texttt{emcee}
\item \texttt{python-FSPS}
\item \texttt{sedpy} (for nonparametric SFHs)
\end{itemize}

\section{User Interaction}
The primary user interaction is through a dictinary and a list that describe a number of parameters.  The dictionary and list can be specified in a json file, or in a python script, whichever is more convenient.

\subsection{Model parameter definitions: The \emph{model_params} list}
Each model parameter is given an entry in a list called {\bf model_params}. Each list element is a dictionary which contains at minimum the keys {\it name, N, init, isfree} with values giving the parameter name, number of elements or length of the parameter (\texttt{1} for a scalar), initial value(s), and a boolean specifying whether the parameter is allowed to vary or not.  Note that the value of {\it init} can be a vector or list if {\it N}$>1$.

For parameters with {\it isfree}=\texttt{True} the following additional keys of the dictionary are required: {\it prior_function_name} and {\it prior_args} with values consisting of a string giving the name of the prior function (e.g. \texttt{``tophat''}) and a dictionary of arguments to the prior function.

It's also a good idea to have a {\it units} key, and maybe a {\it label} key.

All the model parameters will be passed to the model object on initialization.  The free parameters will be varied by the code during the optimization and sampling phases.  The initial value from which optimization is begun is set by the {\it init} values of each parameter.

\subsection{Additonal parameters definition: The \emph{run_params} dictionary}
Beyond {\bf model_params}, the following parameters conrol key aspects of the operation of the code, and a re stored in a special dictionary called {\bf run_params}
\begin{itemize}
\item {\it verbose} - Boolean
\item {\it mock} - optional Boolean
\end{item}

\subsection{Data format: The \emph{obs} dictionary}
The code expects observed values or data to be in a dictionary (here called the {\bf obs} dictionary) with the following keys:
\begin{itemize}
\item {\it wavelengths}
\item {\it spec}
\item {\it unc}
\item {\it mask}
\item {\it filters}
\item {\it mags}
\item {\it mags_unc}
\item {\it phot_mask}
\end{itemize}

You should have a function that takes arguments from the {\bf run_params} dictionary and produces the {\bf obs}  dictionary.


\section{Advanced Usage}
\subsection{Mock data}
Really this should not be advanced.  Everyone should do it

\subsection{User defined models}

\subsection{MPI}

\end{document}

