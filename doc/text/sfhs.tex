\documentclass[12pt, letterpaper, preprint]{aastex} 

\usepackage{graphicx}
\usepackage{amsmath}

\newcommand{\tmin}[1][]{t_{\mathrm{min} #1}}
\newcommand{\tmax}[1][]{t_{\mathrm{max} #1}}
\newcommand{\amin}{a_{\mathrm{min}}}
\newcommand{\amax} {a_{\mathrm{max}}}

\newcommand{\dlt}{\Delta\log t_i}
\newcommand{\dt}{\Delta t_i}

\newcommand{\tintegral}{\int_{\tmin[,i]}^{\tmax[,i]} dt}
\newcommand{\tinterval}{\right|_{\tmin[,i]}^{\tmax[,i]}}
\newcommand{\clip}[3][]{{#1}_{\top {#2}}^{\bot {#3}}}

\newcommand{\sftrunc}{T_{\mathrm{trunc}}}
\newcommand{\tage}{t_{\mathrm{age}}} 
\newcommand{\sfzero}{T_{\mathrm{zero}}}
\newcommand{\tzero}{t_{\circ}}
\newcommand{\sfslope}{m_{\mathrm{SF}}} 


\begin{document}
\author{B. Johnson}

%\begin{center}
%\today
%\end{center}

\section{Introduction}
We wish to find the proper weighting of discrete SSPs to reproduce the spectrum $f$ from SFR occurring within some bin of  lookback times $(\amin, \amax)$.
We approximate the spectrum $f(t)$ at any lookback time $t$ as a linear interpolation of the (total stellar mass normalized) spectra at the discrete SSP lookback times $t_i$, for $i=1,...,N$ 
\begin{eqnarray}
f & = & \int_{\amin}^{\amax} dt \, f(t) \, SFR(t) \nonumber \\
f(t) & = & w_{i}(t) \, f_i + w_{i+1}(t) \, f_{i+1} \, , \quad t_i < t < t_{i+1}  \nonumber \\
w_{i}(t) & = & \frac{t_{i+1} - t}{t_{i+1}  - t_i} \nonumber \\
w_{i+1}(t) & = & 1 - w_i(t) \quad . \nonumber
\end{eqnarray}
Then we can write the \emph{total} contribution of SSP $i$ to the total spectrum as a sum of the integrated weight due to SF in the sub-bin to the ``right'' (higher lookback times) and the  sub-bin to the ``left'' (lower lookback times).
The sub-bins are defined by the spacing of the SSP $t_i$ values. 
The limits of each integral are determined either by the spacing of the SSP points $t_i$ or, when the bin does not completely span the sub-bin, by the $\amin$ and/or $\amax$ values.
\begin{eqnarray}
W_i & = &  \tintegral \, \mathrm{SFR}(t) \, w_i(t) \nonumber \\
 & = & W_{i, \mathrm{right}} + W_{i, \mathrm{left}} \nonumber \\
W_{i, \mathrm{right}} & = & \tintegral \, \mathrm{SFR}(t) \, \frac{t_{i+1} - t}{t_{i+1}  - t_i} \nonumber \\
W_{i, \mathrm{left}} & = & \int_{\tmin[,i-1]}^{\tmax[,i-1]} dt \, \mathrm{SFR}(t) \, \frac{t - t_{i-1}}{t_{i}  - t_{i-1}} \nonumber \\
\tmin[,i] & = & \clip[\amin]{t_i}{t_{i+1}} \nonumber \\
\tmax[,i] & = & \clip[\amax]{t_i}{t_{i+1}} \nonumber
\end{eqnarray}
where $\clip[x]{a}{b}$ denotes $x$ \emph{clipped} to the interval $(a,b)$. If we take $\mathrm{SFR}(t) = 1$ then we can define
\begin{eqnarray}
g_i(x) & \equiv & \tintegral \, \mathrm{SFR}(t) \, \frac{x - t}{t_{i+1}  - t_i} \nonumber  \\
 & = & \left. \frac{1}{t_{i+1} - t_i} \, \left[ x\, t - \frac{t^2}{2} \right] \tinterval
\end{eqnarray}
which leads to
\begin{eqnarray}
W_i & = & g_{i}(t_{i+1}) - g_{i-1}(t_{i-1})
\end{eqnarray}
and one only has to calculate the $\{g_i(t_i)\}, \{g_i(t_{i+1})\}$ once and then take differences between shifted versions to obtain the total weights.
Note that for the smallest $i$ we define $g_{i-1}(t_{i-1})\equiv 0$ (there is no younger or ``left'' sub-bin) 
and for the largest $i$ we define $g_{i}(t_{i+1}) \equiv 0$ (there is no older or ``right'' sub-bin) .

This formalism works even if $\amin$ and $\amax$ both fall between the same $t_i$ and $t_{i+1}$; that is, if the desired bin is narrower than the SSP defined sub-bins.
It also works if a sub-bin falls completely outside the $(amin, amax)$ interval.
This is because the clipping will cause $\tmin$ and $\tmax$ to be the same for bins not at least partially within the $(amin, amax)$ interval, and thus all integrals will evaluate to zero.

\section{Interpolation in $\log t$}

For the case of a linear interpolation in $\log t$, which is probably more appropriate, we can write
\begin{eqnarray}
w_{i}(t) & = & \frac{ \log t_{i+1} - \log t}{ \log t_{i+1}  - \log t_i} \nonumber
\end{eqnarray}
in the first set of equations above. The we simply replace $g_i$ by $s_i$ defined as 
\begin{eqnarray}
s_i(x) & \equiv & \tintegral \, \mathrm{SFR}(t) \, \frac{x - \log t}{\log t_{i+1}  - \log t_i} \nonumber \\
 & = & \left. \frac{1}{\log t_{i+1} - \log t_i} \left[ t \, x- t \log\left(\frac{t}{e}\right) \right] \tinterval
\end{eqnarray}
to obtain a total weight for SSP $i$ of
\begin{eqnarray}
W_i & = & s_{i}(\log t_{i+1}) - s_{i-1}(\log t_{i-1}) \nonumber
\end{eqnarray}

The key in both cases is to properly calculate $\tmax$ and keep track of the indices 
(especially for the first and last SSP, where $s_{i-1}$ and $s_i$ are not defined, respectively.)  

\section{Lookback Time Zero}
The youngest age for which an SSP spectrum is availables is generally of order 1 Myr.  
However, we need to account for SF occuring from 0-1 Myr lookback time.
There's a couple ways to deal with this necessary extrapolation.
Probably the most straightforward is nearest-neighbor extrapolation.  
That is, simply approximate the $t=0$ spectrum by the youngest available SSP.

In the case of logarithmic interpolation one has to define some minimum age, say 1 year, to act as zero.
In practice the exact choice of minimum age does not matter, 
as the total weights for the zeroth and first SSP converge such that fractional error is below $\sim 10^{-4}$ for a minimum age less than 10 years or so...
The totally correct way to do this is to work out the definite integrals properly with $\tmin[,i]=0$.


\section{More complex SFHs}
For more complicated SFH only the integrals in the basic definition of $s_i$ and $g_i$ must be calculated again by hand, 
(remembering here that $t$ is \emph{lookback time}.)
We define $s_i(x)$ for interpolation in $\log t$ and $g_i(x)$ for interpolation in $t$.  
For simplicity and clarity we take the cases $x = \log t_{i+1}$ for $s_i$ and $x = t_{i+1}$ for $g_i$, 
but substitution for $x$ is trivial (as long as it is not $t$ dependent):
\begin{eqnarray}
s_i & \equiv & \tintegral \, \mathrm{SFR}(t) \, \frac{\log t_{i+1} - \log t}{\dlt} \\
g_i & \equiv & \tintegral \, \mathrm{SFR}(t) \, \frac{t_{i+1} - t}{\dt}
\end{eqnarray}
where for brevity we've defined in this section
\begin{eqnarray}
\dt & \equiv & t_{i+1} -  t_i  \nonumber \\
\dlt & \equiv & \log t_{i+1} - \log t_i \nonumber
\end{eqnarray}
In general the limits of the integral ($\tmin, \tmax$) will be defined by the SSP temporal grid $\{t_i\}$ as well as any discontinuities in the derivative of the SFR (e.g. the edges of step function already enocountered).

For convenience we will also note that the integral
\begin{eqnarray}
H_i & \equiv & \tintegral \, e^{t/\tau} \, \log t \nonumber \\
  & = & \left. \tau \left[ \log t \, e^{t/\tau} - \log(e) \, \mathrm{Ei}(t/\tau) \right]\tinterval
\end{eqnarray}
where Ei is the exponential integral.

\subsection{$\tau$-model}
Here we have a SFH defined by three parameters
\[ 
\mathrm{SFR}(t) = 
\begin{cases}
K \, e^{-T/\tau}, &  \text{if } 0 < T \leq \sftrunc \\
0, & \text{otherwise}
\end{cases}
\]
\begin{eqnarray}
T & = & \tage - t \nonumber \\
t_q & = & \tage - \sftrunc \nonumber 
\end{eqnarray}
where $T$ is defined as the time since the start of star formation, \emph{not} lookback time.  
The lookback time  (in the reference frame of the galaxy) is given by $t$, which goes from 0 to $\tage$.
The parameters are: 
$\tage$, the age of the galaxy since the start of SF (\texttt{tage} in FSPS), 
$\tau$, the $e$-folding time (\texttt{tau} in FSPS), and
$\sftrunc$, the time since the start of SF that truncation occurs (\texttt{sf\_trunc} in FSPS).
Note that in the current FSPS definition, $\sftrunc = 0$ corresponds to no truncation, so it is actually $\sftrunc= \tage$ and $t_q=0$.
The FSPS \texttt{sf\_start} is basically redundant with \texttt{tage} and confusing so we ignore it.

In this case we have, for the logarithmic time interpolation,
\begin{eqnarray}
s_i  & = & \tintegral \, K \, e^{(t-\tage) /\tau}\, \frac{\log t_{i+1} - \log t}{\dlt} \nonumber \\
      & = & \frac{K \, e^{-\tage/\tau}}{\dlt} \left[ \left. \tau \, \log t_{i+1} \, e^{t/\tau} \tinterval -  H_i\right] \nonumber \\
\tmin[,i] & = & \clip[t_q]{t_i}{t_{i+1}} \nonumber \\
\tmax[,i] & = & \clip[\tage]{t_i}{t_{i+1}} \quad . \nonumber
\end{eqnarray}
Substituting for $H_i$ we have, finally, 
\begin{eqnarray}
s_i & = & \left. \frac{K \, \tau \, e^{-\tage/\tau}}{\dlt} \left [(\log t_{i+1} - \log t)\, e^{t/\tau} + 
          \log(e) \, \mathrm{Ei}(t/\tau)\right]  \tinterval
\end{eqnarray}
For interpolation in linear time the integral is a bit more straightforward
\begin{eqnarray}
g_i & = & \left. \frac{K \, \tau \, e^{-\tage/\tau}}{\dt} (t_{i+1} - t + \tau)\, e^{t/\tau} \tinterval
\end{eqnarray}
where the limits $\tmin$ and $\tmax$ are defined as above.
One can either calculate the normalization constant $K$ analytically, or simply renormalize the final weights $W_i$ by their sum.

\subsection{Delayed-$\tau$}
Here the SFH is also defined by three parameters
\[ 
\mathrm{SFR}(t) = 
\begin{cases}
K \, T \, e^{-T/\tau}, &  \text{if } 0 < T \leq \sftrunc\\
0, & \text{otherwise}
\end{cases}
\]
\begin{eqnarray}
T & = & \tage - t \nonumber \\
t_q & = & \tage - \sftrunc\nonumber
\end{eqnarray}
So now we have
\begin{eqnarray}
s_i  & = & \tintegral \, K \, (\tage-t) \, e^{(t-\tage) /\tau}\, \frac{\log t_{i+1} - \log t}{\dlt}  \nonumber \\
      & = & \frac{K \, e^{-\tage/\tau}}{\dlt} \tintegral \, \left(\tage \, \log t_{i+1} \, e^{t/\tau}  - \tage \, \log t \, e^{t/\tau} - t \, \log t_{i+1} \, e^{t/\tau} + t \, \log t \, e^{t/\tau}\right) \quad .\nonumber
\end{eqnarray}
The first and third terms are fairly straightforward.  The second term is basically the integral we did for the $\tau$-model.  The last term can be solved by integrating by parts (with $u=t\,\log t$, $dv=e^{t/\tau}dt$ and hence $du=(\log e + \log t)\, dt$, $v=\tau e^{t/\tau}$.  Substituting and collecting terms gives, finally, 
\begin{eqnarray}
s_i  & = & \frac{K \, \tau \,  e^{-\tage/\tau}}{\dlt} \left[\left.\left(t\,\log \frac{t}{t_{i+1}} + \tage\,\log t_{i+1}  + \tau\,\log \frac{t_{i+1}}{e}\right) e^{t/\tau}\tinterval - (\tage/\tau+1)\, H_i\right] \quad .
\end{eqnarray}
The limits $\tmin$ and $\tmax$ are as for the $\tau$-model above.

For interpolation in linear time, we have
\begin{eqnarray}
g_i  & = & \tintegral \, K \, (\tage-t) \, e^{(t-\tage) /\tau}\, \frac{t_{i+1} - t}{\dt}  \nonumber \\
      & = & \frac{K \, e^{-\tage/\tau}}{\dt} \tintegral \, \left(\tage \, t_{i+1} \, e^{t/\tau}  - \tage \, t \, e^{t/\tau} - t \,  t_{i+1} \, e^{t/\tau} + t^2 \, e^{t/\tau}\right) \nonumber \\
     & = & \frac{K \, \tau \, e^{-\tage/\tau}}{\dt} \left. e^{t/\tau} \, \left[ \tage \, t_{i+1} - (\tage + t_{i+1})\, (t-\tau) + t^2 - 2\, t\, \tau + 2\, \tau^2\right] \tinterval \quad .
\end{eqnarray}

One could, in principle, redefine the SSP $t_i$ to be in forward time instead of lookback time.  While this would simplify the integrals, keeping them in lookback time is conceptually simpler.

\subsection{Simha}
This is the SFH from \citet{simha14}.  
It is defined by a delayed-$\tau$ SFH until some time $\sftrunc$, after which the SFH is linearly increasing or decreasing with some slope $m_{\mathrm{SF}}$.  
The SFR is continuous at $\sftrunc$, but the derivative is not.  
For this reason, we will treat the SFH in two separate pieces.  
The trick is to get the proper normalization at $\sftrunc$ and, for linearly decreasing SF, find the time where the SF reaches zero. 
\[ 
\mathrm{SFR}(t) = 
\begin{cases}
K \, T \, e^{-T/\tau}, &  \text{if } 0 < T \leq \sftrunc \\
L\, \left[1 - \sfslope \, (T - \sftrunc)\right] & \text{if } \sftrunc < T  < \sfzero \\
0 & \text{otherwise}
\end{cases}
\]
\begin{eqnarray}
T & = & \tage - t \nonumber \\
t_q & = & \tage - \sftrunc \nonumber \\
L & = & K \, \sftrunc \, e^{-\sftrunc/\tau} \nonumber \\
\sfzero & = & \clip[\left(\sftrunc + \sfslope^{-1}\right)]{\sftrunc}{\tage}, \nonumber \\
\tzero & = & \clip[\left( t_q - \sfslope^{-1}\right)]{0}{t_q} \nonumber
\end{eqnarray}
where $\tzero$ is the \emph{lookback} time at which the SFR becomes zero.

The portion of the SFH that is a delayed-$\tau$ can be dealt with using the formulae in the last section.  The linear portion is relatively straightforward:
\begin{eqnarray}
s_i  & = & \tintegral \,  L\, \left[1 - \sfslope \, (t_q - t)\right] \, \frac{\log t_{i+1} - \log t}{\dlt}  \nonumber \\
      & = & \frac{L}{\dlt} \left.\left[(1-\sfslope\, t_q) \, t \, \left(\log t_{i+1} - \log\frac{t}{e} \right) + \frac{\sfslope \, t^2}{2} \, \left(\log t_{i+1} - \log t +\frac{\log e}{2}\right)  \right] \tinterval \\
\tmin[,i] & = & \clip[\tzero]{t_i}{t_{i+1}} \nonumber \\
\tmax[,i] & = & \clip[t_q]{t_i}{t_{i+1}} \quad . \nonumber
\end{eqnarray}

The linear interpolation version of the linear SFH is given by 
\begin{eqnarray}
g_i  & = & \tintegral \,  L\, \left[1 - \sfslope \, (t_q - t)\right] \, \frac{t_{i+1} - t}{\dt}  \nonumber \\
      & = & \frac{L}{\dt} \left.\left[(1-\sfslope\, t_q) \, t_{i+1} \, t + \left(\sfslope\, t_{i+1} + \sfslope \, t_q -1\right)\, \frac{t^2}{2} - \frac{\sfslope\, t^3}{3}\right] \tinterval
\end{eqnarray}

We verify that these expressions reduce to the constant SFR case when $\sfslope=0$.

\subsection{Bursts}

Here we have 
\[ 
\mathrm{SFR}(t) = \delta(T - T_{burst})
\]
\begin{eqnarray}
T & = & \tage - t \nonumber \\
T_{burst} & = & \tage - t_b
\end{eqnarray}

\begin{eqnarray}
T & = & \tage - t \nonumber \\
T_{burst} & = & \tage - t_b
\end{eqnarray}


\end{document}